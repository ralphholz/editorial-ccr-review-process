%%%%%%%%%%%%%%%%%%%%%%%%%%%%%%%%%%%%%%%%%%%%%%%%%%%%%%%%%%%%%%%%%%%%%
%% Title: Editorial
%%%%%%%%%%%%%%%%%%%%%%%%%%%%%%%%%%%%%%%%%%%%%%%%%%%%%%%%%%%%%%%%%%%%%

\documentclass[letterpaper]{article}
\usepackage{background}
\usepackage{geometry}
\geometry{
  letterpaper,top=3cm,
  bottom=4cm,left=3cm,right=4cm 
}
\usepackage{fancyhdr}
\usepackage{soul}
\usepackage{url}
\usepackage{xcolor}



\SetBgAngle{0}
\SetBgColor{black}
\SetBgContents{\rule[0.6ex]{.5pt}{\paperheight}}
\SetBgPosition{current page.west}
\SetBgHshift{+2cm}

\newcommand{\etal}{\textit{et al}. }
%%%%%%%%%%%%%%%%%%%%%%%%%%%%%%%%%%%%%%%%%%%%%%%%%%%%%%%%%%%%%%%%%%%%%
%                      YOUR INFORMATION
%
%      PLEASE EDIT THE FOLLOWING LINES ACCORDINGLY!!
%%%%%%%%%%%%%%%%%%%%%%%%%%%%%%%%%%%%%%%%%%%%%%%%%%%%%%%%%%%%%%%%%%%%%
\newcommand{\review}{Update on ACM SIGCOMM CCR reviewing process: making the review process more open}
\newcommand{\authors}{CCR Editorial board} 
%\newcommand{\reviewer}{Steve Uhlig}
%\newcommand{\revieweraff}{CCR Editor}
\newcommand{\issue}{3}
\newcommand{\volume}{50}
\newcommand{\pubmonth}{July}
\newcommand{\pubyear}{2020}


\fancypagestyle{plain}{%
   \fancyhf{} %ls
   \fancyfoot[L]{\vspace{1cm} ACM SIGCOMM Computer Communication Review }%
   \fancyfoot[R]{\vspace{1cm} Volume \volume{ }Issue \issue,
     \pubmonth{ }\pubyear}%
}
%\renewcommand{\headrulewidth}{0.4pt}% Default \headrulewidth is 0.4pt 
%\renewcommand{\footrulewidth}{0.4pt}% Default \footrulewidth is 0pt 

\pagestyle{plain}
\def\refname{}


\usepackage{multicol}
\setlength{\columnsep}{1cm}
 
\begin{document}
\begin{multicols}{2}
[
%\begin{center}{ {\normalfont\sffamily\Large\bfseries {Public Review for}}} \end{center}
\begin{center}{ \normalfont\sffamily\LARGE\bfseries{\review}}\end{center}
\begin{center}{\normalfont\sffamily\large {\authors}}\end{center}
]
\renewcommand{\sfdefault}{ptm}

%%%%%%%%%%%%%%%%%%%%%%%%%%%%%%%%%%%%%%%%%%%%%%%%%%%%%%%%%%%%%%%%%%%%%
%                      Editorial 
%%%%%%%%%%%%%%%%%%%%%%%%%%%%%%%%%%%%%%%%%%%%%%%%%%%%%%%%%%%%%%%%%%%%%

The purpose of this editorial note is to update the SIGCOMM community on the
reviewing process in place currently at CCR, and our plans to evolve it to
make CCR a more open and friendly venue to authors, adding more value to
the community.

\hl{RH: I wonder if we should briefly outline what's to come}

\section{Role of editorial board and turn-around time}

\hl{This section seems to have at least two purposes: introduce the new
communication channels, and then empowering the ACs. I wonder if we should
describe the role of the board first, then the suggested communication
channels, and move the rest to a new section called `Faster turn-arounds:
empowering area chairs'}

\subsection{Role of the board}


At the moment, once a paper is submitted on the CCR hotcrp review system,
the CCR editor assigns the paper to an area chair (based on the area
chair's expertise but also trying to balance the reviewing load across all
area chairs), and the area chair in turn selects the reviewers. Typically,
three reviewers are invited for each paper, and a decision is made within
about 10 weeks from the reviewer assignment\footnote{Note that we also plan
to speed up the turn-around time for returning decisions, especially when
the decision is a reject. This is already done for out-of-scope submissions
through a fast-track reject by the editor (\textbf{should we allow area
editors as well to do it?}).}. The decision is either made solely by the
area chair (when the outcome is clear-cut based on the reviews), or through
a collegial involvement of the reviewers and/or the CCR editor (e.g., often
for ``revise and resubmit to next issue'' decisions). Three types of
decisions are currently employed: accept, reject, and ``revise and resubmit
to the next issue''. \hl{I always thought these decisions are odd. From
experience, even an Accept still means considerable work to work in
reviewer requests. It is not necessarily the light touch requested by a
Minor Revision. Note that journals also distinguish between Major Revision
and Revise-and-resubmit. I realise it has to do with the review concluding
and a decision for inclusion in the next issues being needed asap.} In the
case of an acceptance and rejection, a summary letter is drafted by the
area chair, and posted on the hotcrp system as a comment to the submission.
The case ``revise and resubmit to the next issue'' is more involved, as in
this case the authors have to be able to submit their revisions within a
limited time (typically for the next issue, hence less than 3 months after
the decision) \hl{What is actually the use of that?}. From now on, we plan
to empower area chairs to be solely responsible for making decisions,
though still collegially with the reviewers, but without the involvement of
the CCR editor as much as possible. As will be explained in the next
section, the more open communication channels between area chair,
reviewers, and authors, should make this easier in the future.

\subsection{Communication channels}
\hl{Here, we could write about how we envisage the communication to take place.}


\section{Reviewer anonymity: breaking the anonymity wall}

\hl{Probably so important that we want a section of its own. It is worded a bit into-your-face at the moment, though. Can SIGCOMM members take that? :) Importantly, I would give examples of strong, well-known journals that do employ open peer review. We could also announce the beginning of a trialling period where ACs only ask all reviewers to be visible. We should definitely give positive examples of courteous process and better papers as a result.}

Even though currently the review form already includes a tick box to allow
reviewers to make their name visible to authors, it is not yet widespread
practice for reviewers to tick it. From now on, the default expectation is
that reviewers invited to evaluate a CCR submission will have to be
non-anonymous to authors, unless they invoke a very special reason (what
such reasons could be?). We hope that by doing this, the whole reviewing
process will be more open, and enable direct communication (still through
the hotcrp reviewing platform of CCR) between authors and reviewers, for
example to clarify misunderstandings regarding the paper, for example to
let authors better explain what they believe the contribution of their
paper is, which might not have been understood by the reviewer. Once more,
the next section will elaborate on how changing the reviewing form may
support this.

\section{Paper evaluation: adding value to the community}

\hl{I agree, although I would probably portray this as allowing a more balanced
way to appreciate contributions. It does represent a departure of CCR's
original mission of fast dissemination of new result towards a wider scope.}

Currently, the CCR review form uses novelty and how well written the paper is
as metrics upon which a decision is made whether to accept a paper or not. We
have now changed the review form to also include a numerical score for value to
the research community and reproducibility, which will be taken into account
into the review process to make decisions. Despite how important technical
novelty is for paper acceptance, we believe that the SIGCOMM community
generally places too much emphasis on it. This leads to a paper selection
process biased against very important contributions that add potentially even
more value to the community than technically novel contributions. CCR already
accepts a variety of contributions, such as the editorial notes, reproducible
papers, and recently educational contributions. We believe that opening up to
different types of contributions was already a huge step forward. However,
unless we explicitly allow reviewers to identify and positively score the value
a submission provides to the community, the process is still not as open as it
could be towards contributions whose contribution does not clearly fall within
the pre-defined types CCR currently accepts.


%                      description of papers



%I hope that you will enjoy reading this new issue and welcome 
%comments and suggestions on CCR Online (\url{https://ccronline.sigcomm.org})
%or by email at \texttt{ccr-editor}\emph{ at }\texttt{sigcomm.org}. 



\vspace{1cm}
\hfill
\begin{minipage}{.7\columnwidth}
\begin{flushright}                                      
%\emph{Public review written by}\\
%\textbf{\reviewer}\\
%\emph{\revieweraff}\\
\end{flushright} 
\end{minipage}

\end{multicols}
\end{document}
